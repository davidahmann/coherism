\documentclass[11pt]{article}
\usepackage{jmlr2e}
\usepackage{hyperref}

\begin{document}

\section*{Data and Code Availability Statement}

All code, data, and materials required to reproduce the results in this paper are publicly available.

\subsection*{Code Repository}

The complete implementation is available at:
\begin{center}
\url{https://github.com/davidahmann/coherism/tree/main/alfm_bem}
\end{center}

\subsection*{Contents}

The repository includes:
\begin{itemize}
    \item \textbf{Source code}: Complete implementation of the ALFM-BEM architecture, including:
    \begin{itemize}
        \item Bidirectional Experience Memory (BEM) with risk, success, and coverage signals
        \item Consensus Engine with Trust, Abstain, Escalate, and Query actions
        \item Bounded Adapters with gradient clipping, norm projection, and EMA smoothing
        \item Contrastive Projection layer for high-dimensional embedding spaces
    \end{itemize}
    \item \textbf{Experiments}: Scripts to reproduce all experimental results:
    \begin{itemize}
        \item \texttt{ablation\_study.py}: Multi-seed ablation comparing BEM, RAG, and NEP baselines
        \item \texttt{threshold\_sensitivity.py}: Parameter sensitivity analysis
        \item \texttt{domain\_shift\_experiment.py}: Adapter stability under distribution shift
        \item \texttt{real\_backbone\_experiment.py}: Integration with sentence-transformers
        \item \texttt{healthcare\_simulator.py}: Healthcare claims processing case study
    \end{itemize}
    \item \textbf{Documentation}: Instructions for installation, configuration, and reproducing all figures and tables
\end{itemize}

\subsection*{Reproducibility}

All experiments use fixed random seeds for reproducibility. The multi-seed experiments report mean $\pm$ standard deviation over 5--10 seeds to establish statistical significance. Hardware requirements are modest: experiments run on standard CPU (M2 Pro) or GPU (NVIDIA T4/A100).

\subsection*{License}

This work is licensed under CC-BY 4.0 (Creative Commons Attribution 4.0 International).

\end{document}
