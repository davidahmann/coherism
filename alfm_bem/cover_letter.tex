\documentclass[11pt]{letter}
\usepackage[utf8]{inputenc}
\usepackage{geometry}
\geometry{a4paper, margin=1in}
\usepackage{hyperref}

\signature{David Ahmann\\Independent Researcher\\Toronto, Canada\\\texttt{dahmann@lumyn.cc}}

\begin{document}

\begin{letter}{Editors-in-Chief\\Journal of Machine Learning Research (JMLR)}

\opening{Dear Editors,}

I am pleased to submit the manuscript entitled ``ALFM-BEM: Bidirectional Experience Memory for Continuous Learning in Foundation Model Deployments'' for consideration as a publication in the Journal of Machine Learning Research.

Foundation models are typically deployed as frozen artifacts, creating a fundamental gap: they cannot learn from their deployment experiences. This paper introduces ALFM-BEM, a unified wrapper architecture that enables continuous learning without modifying backbone weights.

Key contributions of this work include:
\begin{itemize}
    \item \textbf{Bidirectional Experience Memory (BEM)}: A unified memory architecture where experiences exist on a continuous outcome spectrum, providing risk signals, success patterns, and out-of-distribution (OOD) detection as emergent properties of a single structure.
    \item \textbf{Consensus Engine with Query Action}: An extension to selective prediction that transforms passive abstention into active learning by requesting specific information when operating outside the experience distribution.
    \item \textbf{Bounded Adapters}: A mechanism for continuous improvement with provable stability guarantees, preventing catastrophic drift during online updates.
    \item \textbf{Empirical Validation}: Rigorous experiments on synthetic data demonstrate effective failure retrieval (F1 $>$ 0.99) and strong OOD detection (AUC $\approx$ 1.0). Furthermore, a realistic healthcare claims case study shows that ALFM-BEM reduces claim rejection rates by 88\% (from 12.5\% to 1.5\%) by learning latent payer rules from binary feedback alone.
\end{itemize}

This work bridges the gap between static model training and dynamic deployment needs, offering a practical infrastructure for safe, adaptive foundation model systems.

I confirm that this manuscript is original, has not been published before, and is not currently under consideration for publication elsewhere.

Thank you for your consideration.

\closing{Sincerely,}

\end{letter}
\end{document}
