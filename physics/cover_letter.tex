\documentclass[11pt]{letter}
\usepackage[utf8]{inputenc}
\usepackage[margin=1in]{geometry}
\usepackage{hyperref}

\signature{David Ahmann\\dahmann@lumyn.cc\\ORCID: 0009-0006-4066-8760\\Independent Researcher, Toronto, Canada}
\address{David Ahmann\\Toronto, Canada\\dahmann@lumyn.cc}
\date{\today}

\begin{document}

\begin{letter}{Editorial Office\\Foundations of Physics\\Springer Nature}

\opening{Dear Editors,}

I am pleased to submit the manuscript entitled ``\textbf{Coherence-Dependent Backreaction in Semiclassical and Analog Gravity: Testable Predictions from Informational Stress}'' for consideration in \textit{Foundations of Physics}.

\textbf{Context and Importance.} The relationship between quantum coherence and spacetime geometry remains one of the most challenging open problems in theoretical physics. While semiclassical gravity treats quantum matter as a source for classical geometry, it does not distinguish between coherent and incoherent quantum states of identical energy density. This paper addresses that gap by deriving an \emph{informational stress tensor} from a variational principle based on relative entropy.

\textbf{Primary Contribution.} The key result is a concrete, falsifiable prediction for analog gravity experiments in Bose-Einstein condensates: coherent phonon injection near a sonic horizon produces density modulations $\delta\rho/\rho_0 \sim 10^{-6}$, while thermal phonon injection of identical energy produces no such effect. This coherent-versus-thermal signature is:
\begin{itemize}
    \item Measurable with current BEC technology at laboratories such as MIT, JILA, and MPQ;
    \item Absent in competing frameworks (Penrose-Di\'{o}si gravitational collapse, stochastic gravity);
    \item Accompanied by a clear falsification criterion: a null result at $\delta\rho/\rho_0 < 10^{-7}$ would rule out the acoustic implementation.
\end{itemize}

\textbf{Why Foundations of Physics?} This manuscript proposes a new physical principle---that quantum coherence relative to geometry-adapted reference states generates measurable stress-energy---and develops it through rigorous derivation while maintaining focus on experimental testability. This combination of foundational inquiry with concrete predictions aligns well with the journal's scope of addressing fundamental questions through both theoretical and empirical lenses.

The manuscript includes explicit derivations for multiple geometries (Rindler, acoustic, Schwarzschild, FRW), two independent derivations of the coupling constant, and a first-principles derivation of the decoherence generator from Unruh-DeWitt detector dynamics.

\textbf{Declarations.} This manuscript has not been published elsewhere and is not under consideration by another journal. All work is original. The author has no competing interests to declare. AI tools (Claude/Anthropic, GPT-4/OpenAI) were used for copy editing and literature synthesis; all scientific content is the author's own.

Thank you for considering this submission.

\closing{Sincerely,}

\end{letter}
\end{document}
